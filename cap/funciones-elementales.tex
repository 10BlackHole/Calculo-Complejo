\chapter{Funciones elementales}
Consideramos aquí varias funciones estudiadas en el Cálculo y definimos funciones correspondientes de una variable.

\section{La función exponencial}
Si una función $f$ de una variable compleja $z=x+iy$ se ha de reducir a la función exponencial usual cuando $z$ es real, debemos exigir que

\begin{equation}\label{22.1}
f(x+i0)=e^x
\end{equation}

para todo número real $x$. Ya que $d(e^x)/dx=e^x$ para todo $x$ real, es natural imponer las siguientes condiciones:

\begin{equation}\label{22.2}
f \textup{ es entera y } f'(z)=f(z) \textup{ para todo } z
\end{equation}

La función 

\begin{equation}\label{22.3}
f(z) = e^x(\cos y + i \sin y)
\end{equation}

con $y$ en radianes, es diferenciable en todos los puntos del plano complejo y $f'(z)=f(z)$. Luego esa función que cumple las condiciones (\ref{22.1}) y (\ref{22.2}). Puede probarse además que es la única función que las satisface; y escribiremos $f(z)=e^x$. 

Así la función exponencial del análisis compeljo se define para todo $z$ como 

\begin{equation}\label{22.4}\marginnote{Función exponencial del análisis complejo}
\boxed{e^z=e^x(\cos y +i\sin y)}
\end{equation}

Esta se reduce a la función exponencial usual del Cálculo cuando $y=0$, es entera, y verifica la fórmula diferencial

\begin{equation}\label{22.5}
\frac{d}{dz}e^z=e^z
\end{equation}

en todo el plano.

Hagamos constar además que cuando $z$ es imaginario puro, (\ref{22.4} se convierte en

\begin{equation}
e^{i\theta}=\cos\theta +i\sin\theta
\end{equation}

La forma más compacta

\begin{equation}\label{22.6}
e^z=e^xe^{iy}
\end{equation}





