\chapter{Series}
\section{Covergencia de sucesiones y series}
Una \textit{sucesión} infinita

\begin{equation}\label{44.1}
z_1, z_2,\dots, z_n,\dots
\end{equation}

de número complejos tiene \textit{límite} $z$ si, para cada número positivo $\epsilon$ existe un número positivo $n_0$ tal que

\begin{equation*}
|z_n-z|<\epsilon\qquad \textup{si}\qquad n>n_0
\end{equation*} 

Geométricamente, esto signifca que para valores suficientemente grandes de $n$, los puntos $z_n$ están a cualquier $\epsilon$ entorno dado de $z$. Como podemos elegir $\epsilon$ todo lo pequeño que queramos, se deduce que los puntos $z_n$ se acercan arbitrariamente a $z$ cuando sus subíndices crecen. Nótese que el valor de $n_0$ necesario dependerá, en general, del valor de $\epsilon$.

La suseción \ref{44.1} puede tener a lo sumo un límite. Esto es, un límite $z$ es único, si existe. Cuando existe límite, se dice que la sucesión \textit{converge} a $z$, y escribimos

\begin{equation*}
\lim_{n\to\infty}z_n=z
\end{equation*}

Si la sucesión no tiene límite, \textit{diverge}.

\begin{teo}\label{teo:44.1}
Supongamos que $z_n=x_n+iy_n\, (n=1, 2, ...)\quad \textup{y}\quad z = x+iy$. Entonces
\begin{equation}\label{44.2}
\lim_{n\to\infty}z_n=z
\end{equation}
si y sólo si
\begin{equation}\label{44.3}
\lim_{n\to\infty}x_n=x\qquad \textup{y}\qquad \lim_{n\to\infty}y_n=y
\end{equation}
\end{teo}

Una \textit{serie} infinita

\begin{equation}\label{44.4}
\sum_{n=1}^{\infty}z_n=z_1+z_2+\cdots z_N + \cdots
\end{equation}

de números complejos \textit{converge} con \textit{suma} $S$ si la sucesión

\begin{equation*}
S_N=\sum_{n=1}^Nz_n=z_1+z_2+\cdots z_N\qquad (N=1, 2, \dots)
\end{equation*}

de \textit{sumas parciales} converge a $S$; escribimos en tal caso $$\sum_{n=1}^\infty z_n=S$$

\begin{teo}\label{teo:44.2}\label{teo:2}
Supongamos que $z_n=x_n+iy_n\quad (n=1,2,\dots)\quad \textup{y}\quad S=X+iY$. En tal caso,
\begin{equation}\label{44.5}
\sum_{n=1}^\infty z_n=S
\end{equation}
si y sólo si
\begin{equation}\label{44.6}
\sum_{n=1}^\infty x_n=X\qquad y \qquad \sum_{n=1}^\infty y_n=Y
\end{equation}
\end{teo}

Recordando el cálculo que el n-ésimo termino de una serie convergente de números reales tiende a cero cuando $n$ tiende a infinito, vemos inmediatamente de los Teoremas \ref{teo:44.1} y \ref{teo:44.2} que lo mismo es cierto para una serie convergente de número complejos. Esto es, \textit{una condición necesaria para la convergencia de la serie \ref{44.4} es que}

\begin{equation}\label{44.9}
\lim_{n\to\infty}z_n=0
\end{equation}

Los términos de una serie convergente de números complejos son, por tanto, acotados. Más precisamente, existe una constante positiva $M$ tal que $|z_n|\leq M$ para cada entero positivo $n$.

Para otra propiedad importante de las series de número complejos, suponemos que la serie \ref{44.4} es \textit{absolutamente convergente}. Es decir, si $z_n=x_n+iy_n$, la serie

\begin{equation*}
\sum_{n=1}^\infty |z_n|=\sum_{n=1}^\infty\sqrt{x_n^2+y_n^2}
\end{equation*}

de número reales  $\sqrt{x_n^2+y_n^2}$ converge. Como

\begin{equation*}
|x_n|\leq\sqrt{x_n^2+y_n^2}\qquad \textup{y}\qquad |y_n|\leq\sqrt{x_n^2+y_n^2}
\end{equation*}

sabemos por el criterio de comparación real que las dos series $$\sum_{n=1}^\infty |x_n|\qquad \textup{y}\qquad \sum_{n=1}^\infty |y_n|$$ convergen. Además, ya que la convergencia absoluta de una serie de números reales implica la convergencia de la propia serie, se sigue que existen números reales $X$ e $Y$ tales que se verifica (\ref{44.6}). De acuerdo con el Teorema \ref{teo:44.2}, le serie (\ref{44.4}) converge, por tanto. Por consiguiente, \textit{la convergencia absoluta de una serie de números complejos implica la convergencia de esa serie}.

Al establecer el hecho de que la suma de una serie de un número $S$ dado, es conveniente con frecuencia definir el resto $\rho_N$ tras $N$ términos:

\begin{equation}\label{44.10}
\rho_N=S-S_N
\end{equation}

Luego, $S=S_N+\rho_N$; y, ya que $|S_n-S|=|\rho_N-0|$, vemos que \textit{una serie converge a un número $S$ si y sólo si la sucesión de restos tiende a cero}. Utilizaremos a menudo esta observación al tratar con \textit{series de potencias}. Son series de la forma

\begin{equation*}
\sum_{n=0}^\infty a_n(z-z_0)^n=a_0+a_1(z-z_0)+a_2(z-z_0)^2+\cdots +a_n(z-z_0)^n+\cdots
\end{equation*}

donde $z_0$ y los coeficientes $a_n$ son constantes complejas, y $z$ cualquier punto en una región prefijada que contenga a $z_0$. En tales series, que involucran a una variable $z$, denotaremos las sumas, sumas parciales y restos por $S(z)$, $S_N(z)$, y $\rho_N(z)$ respectivamente.


\section{Serie de Taylor}
Vamos a enunciar el \textit{teorema de Taylor}, uno de los resultados más importante del capítulo

\begin{teo}
Sea $f$ una función analítica en un disco abierto $|z-z_0|<R_0$, centradi en $z_0$ y de radio $R_0$. Entonces, en todo punto $z$ de este disco, $f(z)$ admite la representación en serie de potencias
\begin{equation}\label{45.1}
f(z)=\sum_{n=0}^\infty a_n(z-z_0)^n\qquad (|z-z_0|<R_0)
\end{equation}
donde 
\begin{equation}\label{45.2}
a_n=\frac{f^{(n)}(z_0)}{n!}\qquad (n=0, 1, 2, \dots)
\end{equation}
Esto es, esa serie de potencias converge a $f(z)$ cuando $|z-z_0|<R_0$
\end{teo}

Este es el desarrollo de $f(z)$ en \textit{serie de Taylor} en torno al punto $z_0$. Es la familiar serie de Taylor del Cálculo, adaptada a funciones de una variable compleja.

Obsérvese además que cuando $f$ es \textit{entera}, el radio $R_0$ del disco puede tomarse arbitrariamente grande. En esas circunstancias, la serie converge a $f(z)$ en todo punto $z$ del plano finito, y la confición de validez se convierte en $|z-z_0|<\infty$. 

De la demostración de este teorema nos encontramos con 

\begin{equation}\label{45.8}
f(z)=\sum_{n=0}^\infty \frac{f^{(n)}(0)}{n!}z^n\qquad (|z|<R_0)
\end{equation}

Este caso especial de la serie (\ref{45.1}) en que $z_0=0$ se llama una \textit{serie de Maclaurin}.

Si se sabe que $f$ es analítica en todos los puntos interiores e un círculo cntrado en $z_0$ queda garantizada la convergencia de la serie de Taylor centrada en $z_0$ hacia el valor $f(z)$ en cada uno de esos puntos $z$; no es necesario ningún cirterio de convergencia. En efecto, de acuerdo con el teorema de Taylor, la serie converge a $f(z)$ dentro del círculo centrado en $z_0$ cuyo radio es la distancia de $z_0$ al punto $z_1$ más próximo en el que $f$ deje de ser analítica.





\section{Series de Laurent}
Si una función $f$ no es analítica en un punto $z_0$ no podemos aplicar el teorema de Taylor en ese punto. No obstante, es posible hallar una representación en serie para $f(z)$ que contenga tanto potencias positivas como negativa de $z-z_0$. Ahora presentamos la teoría de tales representaciones, comenzando por el \textit{teorema de Laurent}.

\begin{teo}
Sea $f$ una función analítica en un dominio anular \\ $R_1<|z-z_0|<R_2$, y sea $C$ cualquier contorno cerrado simple en torno de $z_0$, orientado posotivamente, contenido en ese dominio. Entonces, en todo punto $z$ de ese dominio, $f(z)$ admite la representación en serie
\begin{equation}\label{47.1}
f(z)=\sum_{n=0}^\infty a_n(z-z_0)^n+\sum_{n=1}^\infty \frac{b_n}{(z-z_0)^n}\qquad (R_1<|z-z_0|<R_2)
\end{equation}
donde
\begin{equation}\label{47.2}
a_n=\frac{1}{2\pi i}\int_C\frac{f(z)dz}{(z-z_0)^{n+1}}\qquad (n=0,1,2,\dots)
\end{equation}
y
\begin{equation}\label{47.3}
b_n=\frac{1}{2\pi i}\int_C\frac{f(z)dz}{(z-z_0)^{-n+1}}\qquad (n=1,2,\dots)
\end{equation}
\end{teo}

El desarrollo (\ref{47.1}) se suele escribir

\begin{equation}\label{47.4}
f(z)=\sum_{n=-\infty}^\infty c_n(z-z_0)^n\qquad (R_1<|z-z_0|<R_2)
\end{equation}
donde
\begin{equation}\label{47.5}
c_n=\frac{1}{2\pi i}\int_C\frac{f(z)dz}{(z-z_0)^{n+1}}\qquad (n=0,\pm 1, \pm 2, \dots)
\end{equation}

Cualquiera de las dos formas (\ref{47.1}) o(\ref{47.4}), se llama una \textbf{serie de Laurent}.

Nótese que el integrando en (\ref{47.3}) se puede escribir $f(z)(z-z_0)^{n-1}$. Así pued, es claro que cuando $f$ es analítica en el dico $|z-z_0|<R$, este integrando lo es también. Por tanto, todos los coeficientes $b_n$ son cero, y como

\begin{equation*}
\frac{1}{2\pi i}\int_C\frac{f(z)dz}{(z-z_0)^{n+1}}=\frac{f^{(n)}(z_0)}{n!}\qquad (n=0,1,2,\dots)
\end{equation*}

el desarrollo (\ref{47.1}) se reduce a una serie de Taylor centrada en $z_0$.

Sin embargo, si $f$ no es analítica en $z_0$ pero lo es en le resto del disco \\ $|z-z_0|<R_2$, el radio $R_1$ puede tomarse arbitrariamente pequeño. La representación (\ref{47.1}) es válida entonces para $0<|z-z_0|<R_2$. Análogamente, si $f$ es analítica en todo punto del plano finito exterior al círculo $|z-z_0|=R_1$, la condición de validez es $R_1<|z-z_0|<\infty$.

\section{Conergencia absoluta y uniforme de las series de potencias}
El resto de este capítulo se dedica a diversas propiedades de las series de potencias, o sea, serie del tipo $$\sum_{n=0}^\infty a_n(z-z_0)^n$$

Las presentaremos sólo en el caso especial $z_0=0$. Sus demostraciones en el caso general son escencialmente las mismas y muchos de nuestros resultados se generalizan simplemente sustituyendo $z$ por $z-z_0$. Las generalizaciones que afectana a serie con potencia negativas de $z-z_0$ son asimismo fáciles de obtener.

Recordemos que una serie de número complejos converge \textbf{absolutamente} si la serie de valores absolutos de esos número es convergente. El siguiente teorema se refiere a la convergencia absoluta de las serie de potencias.

\begin{teo}
Si una serie de potencias
\begin{equation}\label{49.1}
\sum_{n=0}^\infty a_nz_1^n
\end{equation}
converge cuando $z=z_1\quad (z_1\neq 0)$, entonces es absolutamente convergente en todo punto $z$ del disco abierto $|z|<|z_1|$ 
\end{teo}

\begin{teo}
Si $z_1$ es un punto interior al círculo de convergencia $|z|=R$ de una serie de potencias
\begin{equation}\label{49.4}
\sum_{n=0}^\infty a_nz^n
\end{equation}
entonces esa serie de uniformemente convergente en el disco cerrado $|z|\leq |z_1|$
\end{teo}

\section{Integración y derivación de serie de potencias}
Ya hemos visto que una serie de potencias 

\begin{equation}\label{50.1}
S(z)=\sum_{n=0}^\infty a_nz^n
\end{equation}

representa una función continua en todo punto interior a su círculo de convergencia. En esta sección se enunciará un teorema que nos dice que la suma $S(z)$ es analítica dentro del cículo.

\begin{teo}\label{teo:50.1}
Sea $C$ cualquier entorno interior al círculo de convergencia de la serie de potencias (\ref{50.1}), y sea $g(z)$ cualquier función continua sobre $C$. La serie formada multiplicando cada término de la serie de potencias por $g(z)$ puede ser integrada término a término sobre $C$; esto es.
\begin{equation}\label{50.2}
\int_Cg(z)S(z)dz=\sum_{n=0}^\infty a_n\int_Cg(z)z^ndz
\end{equation}
\end{teo}

Presentamos ahora un resultado el tipo del Teorema \ref{teo:50.1} relativo a la derivación

\begin{teo}
La serie de potencias (\ref{50.1}) puede ser derivada término a término. Esto es, en todo punto $z$ interior al círculo de convergencia de esa serie
\begin{equation}\label{50.6}
S'(z)=\sum_{n=1}^\infty na_n^{n-1}
\end{equation}
\end{teo}

\section{Unicidad de las representaciones por series}
Consideremos en primer lugar la de las serie de Taylor

\begin{teo}
Si una serie
\begin{equation}\label{51.1}
\sum_{n=0}^\infty a_n(z-z_0)^n
\end{equation}
converge a $f(z)$ en todo punto interior a algún círculo $z-z_0=R$, entonces es la serie de Taylor de $f$ en potencias de $z-z_0$.
\end{teo}

Nuestro segundo teorema se refiere a la unciidad de la representación en serie de Laurent

\begin{teo}
Si una serie
\begin{equation}\label{51.3}
\sum_{-\infty}^\infty c_n(z-z_0)^n=\sum_{n=0}^\infty a_n(z-z_0)^n + \sum_{n=1}^\infty \frac{b_n}{(z-z_0)^n}
\end{equation}
converge a $f(z)$ en todos los puntos de algún dominio anular centrado en $z_0$, entonces es la serie de Laurent para $f$ en potencias de $z-z_0$ en ese dominio.
\end{teo}

\section{Multiplicación y división de serie de potencias}
Supongamos que cada una de las series de potencias

\begin{equation}\label{52.1}
\sum_{n=0}^\infty a_nz^n\qquad \textup{y}\qquad \sum_{n=0}^\infty b_nz^n
\end{equation}

converge dentro de un círculo $|z|=R$. Las sumas $f(z)$ y $g(z)$ son funciones analíticas en el disco $|z|<R$, y el producto de esas sumas tiene un desarrollo en serie de Maclaurin válido allí:

\begin{equation}\label{52.2}
f(z)g(z)=\sum_{n=0}^\infty c_nz^n\qquad (|z|<R)
\end{equation}

Como las serie (\ref{52.1}) son las serie de Maclaurin de $f$ y $g$, los tres primeros coeficientes del desarrollo (\ref{52.2}) vienen dados por

\begin{equation*}
c_0=f(0)g(0)=a_0b_0
\end{equation*}
\begin{equation*}
c_1=\frac{f(0)g'(0)+f'(0)g(0)}{1!}=a_0b_1+a_1b_0
\end{equation*}
y
\begin{equation*}
c_2=\frac{f(0)g''(0)+2f'(0)g'(0)+f''(0)g(0)}{2!}=a_0b_2+a_1b_1+a_2b_0
\end{equation*}

La expresión general de $c_n$ es, entonces

\begin{equation}\label{52.3}
[f(z)g(z)]^{(n)}=\sum_{k=0}^n{n\choose k}f^{(k)}(z)g^{(n-k)}(z)
\end{equation}

donde

\begin{equation*}
{n\choose k}=\frac{n!}{k!(n-k)!}
\qquad (k=0,1,2,\dots ,n)
\end{equation*}

para la derivada $n$-ésima del producto de dos funciones. Como es habitual, $f^{(0)}(z)$ y $0!=1$. Evidentemente

\begin{equation*}
c_n=\sum_{k=0}^n\frac{f^{(k)}(0)}{k!}\frac{g^{(n-k)}(0)}{(n-k)!}=\sum_{k=0}^na_kb_{n-k}
\end{equation*}

y así se puede escribir (\ref{52.2}) en la forma

\begin{equation}\label{52.4}
f(z)g(z)=a_0b_0+(a_0b_1+a_1b_0)z+(a_0b_2+a_1b_1+a_2b_0)z^2+\cdots + \left(\sum_{k=0}^na_kb_{n-k}\right)z^n+\cdots \qquad (|z|<R)
\end{equation}

La serie (\ref{52.4}) es la misma que se obtiene multiplicando formalmente las dos series (\ref{52.1}) término a término y reuniendo los términos resultantes por potencias de $z$; esto se conoce como el \textbf{producto de Cauchy} de las dos series dadas.


















